% book example for classicthesis.sty
\documentclass[11pt,a4paper,footinclude=true,headinclude=true]{report} % KOMA-Script book
\synctex=1
\usepackage[T1]{fontenc}
\usepackage{lipsum}
\usepackage[linedheaders,parts,pdfspacing]{classicthesis} % ,manychapters
%\usepackage[osf]{libertine}
\usepackage{datetime}
\usepackage{amsthm}
\usepackage[utf8]{inputenc}
\usepackage[swedish]{babel}
\usepackage[style=verbose,backend=biber,natbib=true]{biblatex}
\usepackage{graphicx}
\graphicspath{ {/home/anon/Dropbox/akademisktskrivande/images/} }
\addbibresource{/home/anon/Dropbox/klassiskmodern/bibliography.bib}
\usepackage{atbegshi} % ta bort blanka sidor



\begin{document}
\thispagestyle{empty}
\begin{center}
\begin{LARGE}
Akademiskt skrivande
\end{LARGE}

\bigskip
En praktisk guide

\bigskip
\bigskip
\bigskip
\bigskip
\bigskip
\bigskip


\includegraphics[scale=.351]{akademisktskrivande.jpeg}
\bigskip
\\


%\begin{tiny}
%\noindent Linnæus, C. (1735) \emph{Systema naturæ, sive regna tria naturæ %systematice proposita per classes, ordines, genera, \& species}, Lugduni %Batavorum: Haak.
%\end{tiny}
\end{center}

\bigskip
\bigskip
\bigskip

\begin{scriptsize}
\noindent Christopher Kullenberg

\noindent Senaste versionen: \href{https://github.com/christopherkullenberg/akademisktskrivande}{https://github.com/christopherkullenberg/akademisktskrivande}
\noindent Denna kompilering: \today
\\
\noindent Licens: GNU General Public License v3.0 
\end{scriptsize}

\newpage %ny sida
\AtBeginShipoutNext{\AtBeginShipoutDiscard}% Discard next blank page

	\tableofcontents
	\thispagestyle{empty}

\newpage %ny sida
\AtBeginShipoutNext{\AtBeginShipoutDiscard}% Discard next blank page
%




\chapter{Referenser och röstseparering}

Videoversion: \href{https://youtu.be/lJbH-CLiLY0}{https://youtu.be/lJbH-CLiLY0}

\section{Refererande i allmänhet}

När man skriver en akademiskt text är man i princip alltid beroende av vad andra har sagt före en själv. Det kan handla om andra forskares resultat, metoder, teorier eller källor hämtade från arkiv, databaser eller från internet. Varje gång man hänvisar till någon annans arbete \textbf{måste} man göra en korrekt referens. Det finns flera anledningar till varför så är fallet:

\begin{small}
\begin{enumerate}
\item Det måste vara tydligt vilka tidigare arbeten du bygger på.
\item Det måste framgå vilka tankar och resultat som är dina och vilka som  är andras. 
\item Undvika plagiat.
\item Det är god sed att referera som en service till läsaren så att det blir enkelt att granska och bygga vidare på texten. 
\end{enumerate}
\end{small}

\noindent Finns det situationer när man inte behöver referera? Svaret är ja i följande två undantag: För det första behöver man inte referera när man skriver fram sina egna tankar eller resultat (eftersom man själv står som författare till texten). Exempelvis:

\begin{quote}
``Jag observerade två kattugglor i Renströmsparken den 16 oktober 2022''

``Jag anser att Einsteins fysik är överlägsen Newtons''
\end{quote}

\noindent I båda dessa fall har jag tydligt angett att jag i det första exemplet själv har genomfört observationen och att det i andra fallet uttrycks en personlig åsikt. 

För det andra behövs det heller ingen referens när man uttrycker något som är allmänt känt, exempelvis:

\begin{quote}
``Polonium upptäcktes av Marie Curie.''

``Kokpunkten för vatten är 100 grader Celsius vid havsnivå.''
\end{quote}

\noindent Vad som är allmänt känt varierar dock beroende på vem som är den tänkta läsaren. En fysiker, en matematiker och en historiker kommer att som läsare betrakta olika saker som allmängods. Som tänkta läsare kommer de även ha olika krav på hur noga och när man bör ange en referens. 

\section{Vad är en korrekt referens?}

Den grundläggande och överordnade principen för att referera är att man ska kunna hitta den källa som åberopas. Exakt hur och med vilken stil man gör det är mindre viktigt, bara man är konsekvent och håller sig till samma stil genom hela texten. 

I löpande brödtext är de två vanligaste sätten att man antingen använder namn-årtal (Svensson, 2022) eller att man gör en fotnot. Vid direkta citat anger man även sidnummer, och ibland ett omfång av sidor när man hänvisar till en viss del av verket. 

I slutet av varje text, eller i vissa fall i fotnoter, så gör man en litteraturlista. Här ska allt man behöver för att hitta källan finnas. Grundläggande för böcker är:

\begin{small}
\begin{itemize}
\item Författare
\item Årtal
\item Titel
\item Förlag
\item Utgivningsort
\end{itemize}
\end{small}

\noindent Därefter har olika källor ytterligare minimikrav:

\begin{small}
\begin{itemize}
\item Tidskriftsartiklar behöver även volym och nummer, samt gärna även DOI-nummer.
\item Internetkällor behöver förutom URL även datum för när källan hämtades.
\item  Databaser, kod-repositories och outgivna källor kan ha sina specifika sätt att hänvisa till. Här får man läsa på från fall till fall.
\end{itemize}
\end{small}

\noindent Det rekommenderas varmt att använda en referenshanterare för att hålla ordning på referenserna och snabbare kunna producera sorterade referenslistor. Det finns flera varianter, både de som är gratis och som kostar pengar, öppen eller stängd källkod och klient- respektive webbaserade tjänster. 

\section{Röstseparering och referenser}

En text består egenligen inte av olika ``röster'', det är ju trots allt fråga om det skrivna ordet snarare än det talade eller sjungna. Men det kan ändå vara användbart att tänka i termer av att en välskriven akademisk text innehåller inte bara många röster, utan även har ordnat dessa tydligt och med en slags harmoni som gör att man enkelt kan följa röst för röst genom texten. 

Den enklaste formen av röstseparering kan exemplifieras med:

\begin{quote}
``Ptolemaios hade en geocentrisk världsbild, medan Kopernikus förespråkade en heliocentrisk modell.''
\end{quote}

\noindent Detta ger oss två röster som är tydligt separerade. Man kan även se det som tre röster, eftersom jag som författare till meningen ju är den som fogat samman Ptolemaios och Kopernikus. 

Vi kan lägga till ytterligare en röst:

\begin{quote}
\textcolor{red}{Ptolemaios} hade en geocentrisk världsbild, medan \textcolor{red}{Kopernikus} förespråkade en heliocentrisk modell. Skiftet mellan dessa två världsbilder kallade \textcolor{red}{Thomas Kuhn} för vetenskapliga revolutioner \footcite{kuhnVetenskapligaRevolutionernasStruktur1979}.
\end{quote}

\noindent I fallen med Ptolemaios och Kopernikus kan dessa påståenden betraktas som allmängods, de behöver ingen direkt referens till originalverken. Men när det kommer till Kuhns begrepp är det både tydligare och mera korrekt att ge en exakt referens till boken \emph{De vetenskapliga revolutionernas struktur}.

Om vi däremot citerar direkt ur en källa måste vi alltid ange en full referens inklusive vilken eller vilka sidor vi har hämtat citatet ifrån:

\begin{quote}
\begin{footnotesize}
Med dagens mått mätt förefaller metodkapitlet i \textcolor{red}{\emph{Systema Naturæ}} vara både kortfattat och kryptiskt. Ur den engelska översättningen från 1806 lyder det i sin helhet: 
\\ 
``[T]he soul of Science, indicates that every natural body may, by inspection, be known by its own peculiar name; and this name points out whatever the industry of man has been able to discover concerning it: so that amidst the greatest apparent confusion, the greatest order is visible.'' \footcite[s. 3]{linneGeneralSystemNature1806}
\end{footnotesize}
\end{quote}

\noindent Vid direkta citat måste vi vara nogranna, speciellt när det rör sig om källor som har utkommit i flera olika utgåvor och översättningar. I detta fall har vi tydligt angett att det är den specifika engelska översättningen från 1806 som vi citerar ur, så att den som granskar citatet kan gå tillbaka till originalet. Två röster förekommer, dels min korta kommentar om att metodkapitlet förefaller vara ``kortfattat och kryptiskt'' och sedan kommer Linnés originalröst (om än översatt till engelska).


\section{Bättre text genom referenser och röstseparering}

Att göra korrekta referenser är enkelt, men kanske mördande tråkigt. Röstseparering är lite svårare, men man får snart till det rätt om man övar lite. Det är alltså ganska små saker, men de har mycket stora effekter på texten som helhet. 

För det första blir den akademiska texten mera korrekt i formell mening. Detta på grund av att texten nu åberopar andra texter på ett sätt som gör att den kan dra nytta av tidigare forskning. En text med många referenser är ofta en text som vilar på en solid grund av inläsning och förståelse för det problem man skriver om. 

För det andra ligger det i det analytiska tänkandets struktur att tänka i termer av ``röster''. Det är när vi jämför, kontrasterar och diskuterar flera olika idéer, resultat eller metoder, som vi gör det möjligt att tänka nytt och självständigt. 

Avslutningsvis kan vi ta ett lite svårare exempel. Gilles Deleuze skrev 1990 en essä med titeln \textit{Postskriptum om kontrollsamhällena}. Den innehåller flera röster som går att färgkoda enligt följande: 

\begin{quote}
Kontrollsamhällena håller på att ersätta de disciplinära samhällena. \textcolor{red}{»Kontroll« är det ord som Burroughs använder för att beteckna detta nya monster}, och som \textcolor{orange}{Foucault utpekade som vår nära framtid}. Även \textcolor{Plum}{Paul Virilio har under lång tid analyserat de ultrasnabba formerna for kontroll i det fria, vilka ersätter de gamla disciplinerna som opererar inom der slutna systemets tid}. Vi behöver inte åberopa de extraordinära nya farmaceutiska produkterna, de nukleära teknologierna eller de genetiska manipulationerna, även om också de kommer att intervenera i denna nya process. Inte heller finns det anledning att fråga vilken regim som är hårdast eller mest uthärdlig, ty i båda två finns en motsats mellan frihet och underkastelse. \footcite[s. 185-6]{deleuzeNomadologin1998}
\end{quote}

I detta stycke finner vi alltså fyra röster: Deleuze själv (svart), Burroughs (röd), Foucault (orange) och Virilio (lila). Denna tydliga röstseparering skapar en komprimerad röd tråd som samtidigt ger en riktning åt författarens tankebana. Men det saknas en viktig detalj för att detta stycke ska uppfylla kriterierna för akademiskt skrivande. Trots att rösterna är välseparerade så saknas de detaljerade referenserna. Var någonstans sade Burroughs\footcite{burroughsElectronicRevolution2001}, Foucault och Virilio\footcite{virilioSpeedPolitics2006} det som de sade? Läsaren måste själv göra detta arbete, och hoppas på att rätt verk har kunnat hittas. 



%\printbibliography
\printbibliography[heading=subbibliography]














    
    \chapter{Röd tråd och stringens}

\noindent När man skriver akademiskt text är det viktigt att man följer en röd tråd i sitt skrivande. Detta kallas ibland för stringens. Det finns tre skäl till varför detta är viktigt.\\
\indent För det första hjälper en röd tråd dig att tänka mera klart och tydligt. Ibland är man redan innan man formulerar sig i skrift inte helt och hållet  klar över vad man vill säga. Då kan man använda sig av en röd tråd för att använda sig av textens ordnande funktion för att hjälpa tankens mera fluktuerande struktur.\\
\indent För det andra bör man alltid skriva så tydligt och logiskt som möjligt för att göra det lättare för en läsare eller granskare av din text att förstå och bedöma det du skriver. \\
\indent För det tredje gör man sig själv en tjänst i det framtida arbetet med texten, från det att man har skrivit ett första utkast till att man reviderar och redigerar fram ett färdigt manuskript. En rörig text är mycket svårare att förbättra än en som redan har en tydlig struktur. Alltså, man tjänar mycket tid och energi på att följa en röd tråd redan från början.







\section{Delar och helheten}


\section{Stycken och rubriker}

\section{Disposition och teckengräns}

\section{Typer av röda trådar}
        




\chapter{Vetenskapligt språk}
           
           Metaregel: vetenskapligt språk beror på \textbf{vem} som är läsaren.

\section{Ordval}
talspråk skriftspråk pronomen
\section{Tydlighet}
Tydlighetskriteriet undvika motsägelser 
\section{Exakthet}
exakthet och precision, vikten av definitioner
\section{Stilistik}
vad kännetecknar akademisk stil och vilka risker finns det att frångå den







    \chapter{Revisioner och redigering}
    
    
 

\end{document}